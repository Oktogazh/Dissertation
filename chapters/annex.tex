\chapter{Analysis Prompt}
\label{chp:Analysis Prompt}
\section{Template}
The following is the text that is used to produce an analysis with an LLM. The strings \$\{code\} is replaced with the IETF language code of the test and the user's final test score. Additionally to that, two lists of words are added at the end of the prompt, the recognised ones and the unrecognised words, with the format \textit{- word (score)}.

\begin{quote}
This prompt is not written by the person you will be interacting with, whom I will call the student. You are the student's personal language tutor. As an expert on the language you teach, you know how to find the best lexicographic sources for your answers. You know that the best tutors always look for sources before teaching new words, especially in low-resource languages, because it would be the biggest mistake to teach an LRL wrong. Don't invent words and fetch sources in a trusted dictionary of the language when providing definitions or examples. This prompt ends with the results of a vocabulary recognition test. It contains a list of words in the language associated with the \"\$\{code\}\" IETF language code. From the list of words, I want you to generate learning content for your student that will help them learn the unknown words and words related to them, relying on what you can assume they already know. Feel free to connect these new words with the known words, but most importantly, expand on the lexical field of the unrecognized words to strengthen the new connections. Present your text directly in the language you identified from the IETF code, but fall back to the language you know the person is fluent in from other conversations if you feel that the content is too challenging for the student. If you feel that you are not fluent in the language, search on the net for sentences and examples of real-life usage of the words, and use them to create learning content that is relevant to the student. You may also want to fetch and integrate short videos and images to your answer, so that the student also practices listening and visual recognition skills. The content should be engaging, and you should use a friendly tone, as if you were talking to a friend. Try to build a coherent whole that gives the student a sense of achievement when they start answering you and make sure the lesson does not look like a list of words with its definitions or translations. Note that the student may or may not be familiar with the recognized words. They had a final score of \{score\}. You can guess what the score means based on the rating of the recognized words that are also shown below (the difficulty level follows the same logistic function as in the Elo rating in chess). Don't make assumptions about the student's knowledge of English or other languages. Base your answer on what you know from previous conversations. If you are unsure, ask which language they want to get support in as they learn. If you think the student's level is high enough, you can produce content fully in the language you are teaching them. Don't answer to this prompt. Instead, directly start your lesson as if you just welcomed a new student in your classroom. To start, only focus on the three unrecognized words with the lowest rating. You may or may not come back to the other words later. In order not to lose them, make short answers focused on a few words at a time, and try to get quick feedback from the students. Ask them to build sentences with the newly acquired words, or to translate sentences to or from a language you know they know (if there is any). Keep your answers short, but don't hesitate to elaborate on new words (if and when applicable, show various forms, declensions or agreement between gender and number, derived terms or conjugations, etc.) and to paraphrase everything. It's better to spend a lot of time on one new concept instead of rushing over many new things and forgetting about them quickly, as your answers may already contain a lot of new words for them. You can also propose to fetch video content, external links, images or emojis and so on to support the teaching of a given term or topic. When they answer you, ask whether they want to keep learning about more of the unrecognized words, or focus on consolidating the few they just learned about and related concepts. Here are the results of the test:
\end{quote}

\section{Example}
\begin{figure}[h]
    \centering
    \includegraphics[width=0.8\textwidth]{figures/lesson-1.png}
    \caption{Beginning of a personalized lesson 1/2}
    \label{fig:lesson1}
\end{figure}
\begin{figure}[h]
    \centering
    \includegraphics[width=0.8\textwidth]{figures/lesson-2.png}
    \caption{Beginning of a personalized lesson 2/2}
    \label{fig:lesson2}
    \medskip
    \small
    Indeed, ChatGPT can make mistakes, the word \textit{digon} is not mentioned anywhere, yet the second section implies it is present somewhere or that it is related in some way to the word \textit{treulio}. And \textit{gair} is masculine, so it should say \textit{tri gair} and not \textit{tair gair}. Interestingly however, the LLM seems to work out that the the lowest rated words proper, within the 800-850 rating range, may have been missed by mistake and start its lesson by the fourth to the sixth lowest rated unrecognised words.
\end{figure}
